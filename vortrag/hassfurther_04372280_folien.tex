\documentclass{beamer}
\usepackage[utf8]{inputenc}
\usepackage[T1]{fontenc}
\usepackage{lmodern}
\usepackage[ngerman]{babel}
\usepackage{hologo}
\usepackage{amsmath}
\usepackage{babelbib}
\usepackage{graphicx}
\usepackage{fancyhdr}
\usepackage{color}
\usepackage{listings}
\usepackage{qtree}
\usepackage{makecell}

\definecolor{GrayCodeBlock}{RGB}{245,245,245}
\definecolor{BlackText}{RGB}{10,10,10}
\definecolor{RedTypename}{RGB}{135,58,3}
\definecolor{GreenString}{RGB}{72,125,44}
\definecolor{PurpleKeyword}{RGB}{159,73,183}
\definecolor{GrayComment}{RGB}{70,70,70}
\definecolor{GoldDocumentation}{RGB}{200,185,65}
\lstdefinelanguage{rust}
{
    escapeinside=``,
    columns=fullflexible,
    keepspaces=true,
    showstringspaces=false,
    frame=single,
    framesep=0pt,
    framerule=0pt,
    framexleftmargin=4pt,
    framexrightmargin=4pt,
    framextopmargin=5pt,
    framexbottommargin=3pt,
    xleftmargin=4pt,
    xrightmargin=4pt,
    backgroundcolor=\color{GrayCodeBlock},
    basicstyle=\ttfamily\color{BlackText},
    keywords={
        true,false,
        unsafe,async,await,move,
        use,pub,crate,super,self,mod,
        struct,enum,fn,const,static,let,mut,ref,type,impl,dyn,trait,where,as,
        break,continue,if,else,while,for,loop,match,return,yield,in
    },
    keywordstyle=\color{PurpleKeyword},
    ndkeywords={
        bool,u8,u16,u32,u64,u128,i8,i16,i32,i64,i128,char,str,
        Self,Option,Some,None,Result,Ok,Err,String,Box,Vec,Rc,Arc,Cell,RefCell,HashMap,BTreeMap,
        macro_rules
    },
    ndkeywordstyle=\color{RedTypename},
    comment=[l][\color{GrayComment}\slshape]{//},
    morecomment=[s][\color{GrayComment}\slshape]{/*}{*/},
    morecomment=[l][\color{GoldDocumentation}\slshape]{///},
    morecomment=[s][\color{GoldDocumentation}\slshape]{/*!}{*/},
    morecomment=[l][\color{GoldDocumentation}\slshape]{//!},
    morecomment=[s][\color{RedTypename}]{\#![}{]},
    morecomment=[s][\color{RedTypename}]{\#[}{]},
    stringstyle=\color{GreenString},
    string=[b]"
}

\lstset{literate=%
    {Ö}{{\"O}}1
    {Ä}{{\"A}}1
    {Ü}{{\"U}}1
    {ß}{{\ss}}1
    {ü}{{\"u}}1
    {ä}{{\"a}}1
    {ö}{{\"o}}1
    {~}{{\textasciitilde}}1
}

\usetheme{Montpellier}

\setbeamertemplate{title page}
{
  \vbox{}
  \begingroup
    \centering
    {
    {\small Bachelorarbeit}\\
    \begin{beamercolorbox}[sep=8pt,center]{title}
      \usebeamerfont{title}\inserttitle\par%
      \ifx\insertsubtitle\@empty%
      \else%
        \vskip0.25em%
        {\usebeamerfont{subtitle}\usebeamercolor[fg]{subtitle}\insertsubtitle\par}%
      \fi%
    \end{beamercolorbox}%
    \usebeamercolor[fg]{titlegraphic}\inserttitlegraphic\par}\vskip1em
    \vskip1em\par
    \begin{beamercolorbox}[sep=8pt,center]{author}
      \usebeamerfont{author}\insertauthor
      \begin{tabular}{ll}
          Verfasser & Bernd Haßfurther \href{mailto:nerglom@posteo.de}{<nerglom@posteo.de>}\\
          Matrikel-Nr. & 4372280\\
          Betreuerin & Prof. Dr. Lena Oden\\
          Datum & \today\\
      \end{tabular}
    \end{beamercolorbox}
    \begin{beamercolorbox}[sep=8pt,center]{institute}
      \usebeamerfont{institute}\insertinstitute
    \end{beamercolorbox}
  \endgroup
  \vfill
}

\addtobeamertemplate{navigation symbols}{}{%
    \usebeamerfont{footline}%
    \usebeamercolor[fg]{footline}%
    \hspace{1em}%
    \insertframenumber/\inserttotalframenumber
}

\setbeamertemplate{section in toc}[sections numbered]
\setbeamertemplate{subsection in toc}[subsections numbered]

\NewDocumentCommand{\codeword}{v}{%
\texttt{\textcolor{blue}{#1}}%
}

\title{Ein Computeralgebrasystem in Rust}
\titlegraphic{\includegraphics[width=2cm]{rust-logo-512x512-blk.png}}

\begin{document}
\maketitle
\frame{
  \frametitle{Inhaltsverzeichnis}
  \tableofcontents
}

\section{Vorstellung Computeralgebrasystem}
\begin{frame}[fragile]
  \frametitle{Was ist ein CAS?}
  \begin{itemize}
    \item Mathematische Ausdrücke mit Variablen darstellen
    \pause
    \item Exakte Darstellung von Ausdrücken
    \pause
    \item Einsatz in verschiedenen Gebieten
    \pause
    \item SymPy als konkrete Implementierung
  \end{itemize}
  {\small (vgl. \cite{FachgruppeDef} \cite[S. 1]{SympyPeerJ})}
\end{frame}

\begin{frame}[fragile]
  \frametitle{Grundregeln und Annahmen}
  \begin{itemize}
    \item Zahlenräume $\mathbb{N}$, $\mathbb{Z}$, $\mathbb{Q}$, $\mathbb{R}$
    \pause
    \item Jede Subtraktion ist eine Addition
    \pause
    \item Jede Division ist entweder eine rationale Zahl oder eine Multiplikation
    \pause
    \item Wurzeln können als Potenzen dargestellt werden
  \end{itemize}
  {\small (vgl. \cite[S. 23 ff.]{Geddes2007} \cite[S. 2]{Tan2000})}
\end{frame}

\begin{frame}[fragile]
  \frametitle{Funktionsumfang der Implementierung}
  \begin{itemize}
    \item Addition von Zahlen und Symbolen
    \pause
    \item Multiplikation von Zahlen und Symbolen
    \pause
    \item Potenzregeln in Hinblick auf Genauigkeit auswerten
    \pause
    \item Auswertung von mathematischen Funktionen und Konstanten
  \end{itemize}
\end{frame}

\section{Vorstellung Rust}
\begin{frame}[fragile]
  \frametitle{Ziele von Rust}
  \begin{itemize}
    \item Performance
    \pause
    \item Verlässlichkeit
    \pause
    \item Produktivität
  \end{itemize}
  {\small (vgl.  \cite{JavaGenerics} \cite[S. 196 ff.]{SK19} \cite{WhyRust})}
\end{frame}

\begin{frame}[fragile]
  \frametitle{Hinweise zur Syntax}
  \begin{itemize}
    \item Expression und Statements \begin{lstlisting}[language=rust]
let t = if bedingung_1 { false } else { true };\end{lstlisting}
    \pause
    \item \codeword{struct} und \codeword{trait} als Klassen und Interfaces \begin{lstlisting}[language=rust]
struct MyStruct {  my_field: i32, }
impl MyStruct { fn do_smth(&self) {} }\end{lstlisting}
    \pause
    \item Enums \begin{lstlisting}[language=rust]
enum MyEnum {
  Entry1(i32, i32, i32), Entry2,
}\end{lstlisting}
  \end{itemize}
  {\small (vgl. \cite[S. 122 ff.]{BO18})}
\end{frame}

\begin{frame}[fragile]
  \frametitle{Hinweise zur Syntax}
  \begin{itemize}
    \item Generics und \codeword{trait objects} \begin{lstlisting}[language=rust]
enum MyEnum<T> { Entry1(T) }\end{lstlisting}
    \pause
    \item Operatorenüberladung \begin{lstlisting}[language=rust]
impl std::ops::Add<MyEnum> for MyEnum {
  fn add(self, rhs: MyEnum) -> MyEnum { ... }
}\end{lstlisting}
    \pause
    \item Referenzen
  \end{itemize}
  {\small (vgl. \cite[S. 246 ff.]{BO18} \cite{TraitBoundariesEx} \cite[S. 196 ff.]{SK19})}
\end{frame}

\begin{frame}[fragile]
  \frametitle{Vertiefung des Ownership und Borrowing}
  \begin{itemize}
    \item Jeder Wert ist einer Variablen zugewiesen, dem \codeword{Owner} 
    \pause
    \item Jeder Wert kann nur einen \codeword{Owner} besitzen
    \pause
    \item Verlässt der \codeword{Owner} den Gültigkeitsbereich, wird die Variable ungültig
    \pause
    \item Beliebige Anzahl an nicht veränderbaren Referenzen oder exakt eine veränderbare Referenz
    \pause
    \item Referenzen müssen immer gültig sein
  \end{itemize}
  {\small (vgl. \cite{Rules1} \cite{Rules2})}
\end{frame}

\begin{frame}[fragile]
  \frametitle{Beispiel Ownership}
  \begin{lstlisting}[language=rust]
let mut a = 2;
let b = a;
a = 4;
println!("{}{}", a, b); // Ausgabe: 42
`\pause`
let s1 = String::from("hello");
let s2 = s1;
println!("{}", s2); // Ok
println!("{}", s1); // Fehler\end{lstlisting}
  {\small (vgl. \cite{Rules1} \cite{Rules2})}
\end{frame}

\begin{frame}[fragile]
  \frametitle{Beispiel Referenzen}
  \begin{lstlisting}[language=rust]
let matrix = LargeMatrix { matrix: vec![] };

fn take_ownership(matrix_fn: LargeMatrix) { ... }
`\pause`
fn give_back_ownership(matrix_fn: LargeMatrix) 
-> LargeMatrix { matrix_fn }
`\pause`
fn take_reference(matrix_ref: &LargeMatrix) { ... }
`\pause`
fn take_mutable_reference(
  matrix_ref: &mut LargeMatrix) { ... }\end{lstlisting}
\end{frame}

\begin{frame}[fragile]
  \frametitle{Beispiel Lifetime}
  \begin{lstlisting}[language=rust]
let r;
{
  let x = 5;
  r = &x;
} 
println!("r: {}", r);
`\pause`
let x = 5;
let r = &x;
println!("r: {}", r);
\end{lstlisting}
{\small (vgl. \cite{LifetimeEx})}
\end{frame}

\section{Implementierung des CAS}
\begin{frame}[fragile]
  \frametitle{Überlauf und Ungenauigkeit}
  \begin{itemize}
    \item Überlauf \begin{lstlisting}[language=rust]
let mut x = i32::MAX;
x += 1;\end{lstlisting}
    \pause
    \item Ungenauigkeiten \begin{lstlisting}[language=rust]
let f1 = 0.1;
let f2 = 0.2;
println!("{}", f1 + f2);\end{lstlisting}
    \pause
    \item Lösung mit externen Abhängigkeiten und Generics
  \end{itemize}
  {\small (vgl. \cite{IEEE754})}
\end{frame}

\begin{frame}[fragile]
  \frametitle{Überlegungen zur Datenstruktur}
  \begin{itemize}
    \item Terme bestehen aus Token
    \pause
    \item Datenstruktur aus Tokens generieren
    \pause
    \item Datenstruktur als Baum
  \end{itemize}
  \Tree[.+
        [.a ]
        [.-1 ]
        [.*
            [.-1 ]
            [.b ]
        ]
    ]
\end{frame}

\begin{frame}[fragile]
  \frametitle{Datenstruktur in Rust}
  \begin{lstlisting}[language=rust]
pub enum Ast<N> {
  Add(Vec<Ast<N>>),
  Mul(Vec<Ast<N>>),
  Pow(Box<Ast<N>>, Box<Ast<N>>),
  Symbol(String),
  Const(String),
  Func(String, Vec<Ast<N>>),
  Num(N),
}\end{lstlisting}
\end{frame}

\begin{frame}[fragile]
  \frametitle{Der Nummerntyp}
  \begin{lstlisting}[language=rust]
pub trait NumberType:
    ...
    + PartialEq<i128>
    + PartialOrd<i128>
    + ops::Add<Self, Output = Self>
    + ops::Mul<Self, Output = Self>
    + From<i128>
{
  fn is_integer(&self) -> bool;
  fn is_float(&self) -> bool;
  fn is_rational(&self) -> bool;
  ...
}\end{lstlisting}
\end{frame}

\begin{frame}[fragile]
  \frametitle{Eine konkrete Implementierung}
  \begin{lstlisting}[language=rust]
pub enum PrimNum {
  Int(i128),
  Float(f64),
  Rational(i128, i128),
}
impl NumberType for PrimNum { ... }

impl From<i128> for PrimNum {
  fn from(from: i128) -> PrimNum {
    PrimNum::Int(from)
  }
}
N::from(1);\end{lstlisting}
\end{frame}

\begin{frame}[fragile]
  \frametitle{Grundfunktionalitäten}
  \begin{itemize}
    \item Implementierung von Operatorenüberladungen
    \pause
    \item Auswertung von Termen
    \pause
    \item Substitution von Variablen
  \end{itemize}
\end{frame}

\begin{frame}[fragile]
  \frametitle{Erweiterung des CAS mit EvalFn}
  \begin{itemize}
    \item Funktionen für Additionen und Multiplikationen
    \pause
    \item Funktionen für Potenzen
    \pause
    \item Möglichkeit für mathematische Funktionen
    \pause
    \item Konstantenauswertung
    \pause
    \item Funktionen um Terme zu vereinfachen
  \end{itemize}
\end{frame}

\begin{frame}[fragile]
  \frametitle{Evaluierung eines Terms}
  \begin{lstlisting}[language=rust]
pub fn eval_sub(
  &self,
  evaler: &EvalFn<N>,
  hard_eval: &bool,
  sub: &Option<&str>,
  with: &Option<&Ast<N>>,
) -> Ast<N> { ... }\end{lstlisting}
\end{frame}

\begin{frame}[fragile]
  \frametitle{Evaluierung eines Terms}
  \begin{lstlisting}[language=rust]
match self {
  Ast::Add(vec) => add(
    vec.iter()
      .map(|t| t.eval_sub(evaler, hard_eval,
        sub, with))
      .collect(),
    evaler,
    hard_eval,
  ),
  ...
  Ast::Symbol(name) if sub.is_some() && 
    name == sub.unwrap() => with.unwrap().clone(),
}\end{lstlisting}
\end{frame}

\begin{frame}[fragile]
  \frametitle{Konkrete Erweiterungen des CAS}
  \begin{itemize}
    \item Additionsterme $cos(x)^2+sin(x)^2 = 1$
    \pause
    \item Potenzen mit Multiplikation als Basis $(4*x)^\frac{1}{2} = 2 * x^\frac{1}{2}$
    \pause
    \item Auswertung der Funktionen $sin$, $cos$ und $limit$ mit Heuristiken
    \pause
    \item Logarithmus-Funktion aufteilen und zusammenfügen $log(x^2*y) = 2*log(x)+log(y)$ und $2*log(x)+log(y) = log(x^2*y)$
  \end{itemize}
  {\small (vgl. \cite{WhyRust})}
\end{frame}

\section{Vergleich zu SymPy}
\begin{frame}[fragile]
  \frametitle{Vergleich zu SymPy}
  \begin{itemize}
    \item Parsing von Termen identisch
    \pause
    \item Unterschiede sind Ausgabedingt
    \pause
    \item Bessere Performance
  \end{itemize}
\end{frame}

\section{Zusammenfassung und Fazit}
\begin{frame}[fragile]
  \frametitle{Verbesserungsideen und deren Ansätze}
  \begin{itemize}
    \item Verbesserung der Performance
    \pause
    \item Verbesserung der Erkennung von Termen
    \pause
    \item Verbesserungen bei der Verarbeitung von Termen
    \pause
    \item Erweiterung der Substituierung
  \end{itemize}
\end{frame}

\begin{frame}[fragile]
  \frametitle{Vor- und Nachteile Rust}
  \begin{itemize}
    \item Ownership- und Borrowing-System anfänglich nicht immer klar
    \pause
    \item Teilweise explizite Codeabschnitte
    \pause
    \item Performance und Fehlerunanfälligkeit
  \end{itemize}
\end{frame}

\section{Quellen}
\begin{frame}[fragile,allowframebreaks]
  \bibliographystyle{babplain-fl}
  \raggedright
  \setbeamertemplate{bibliography item}{\insertbiblabel}
  \bibliography{literature}
\end{frame}
\end{document}