\documentclass{beamer}
\usepackage[utf8]{inputenc}
\usepackage[T1]{fontenc}
\usepackage{lmodern}
\usepackage[ngerman]{babel}
\usepackage{hologo}
\usepackage{amsmath}
\usepackage{babelbib}
\usepackage{graphicx}
\usepackage{fancyhdr}
\usepackage{color}
\usepackage{listings}
\usepackage{qtree}
\usepackage{makecell}

\definecolor{GrayCodeBlock}{RGB}{245,245,245}
\definecolor{BlackText}{RGB}{10,10,10}
\definecolor{RedTypename}{RGB}{135,58,3}
\definecolor{GreenString}{RGB}{72,125,44}
\definecolor{PurpleKeyword}{RGB}{159,73,183}
\definecolor{GrayComment}{RGB}{70,70,70}
\definecolor{GoldDocumentation}{RGB}{200,185,65}
\lstdefinelanguage{rust}
{
    escapeinside=``,
    columns=fullflexible,
    keepspaces=true,
    showstringspaces=false,
    frame=single,
    framesep=0pt,
    framerule=0pt,
    framexleftmargin=4pt,
    framexrightmargin=4pt,
    framextopmargin=5pt,
    framexbottommargin=3pt,
    xleftmargin=4pt,
    xrightmargin=4pt,
    backgroundcolor=\color{GrayCodeBlock},
    basicstyle=\ttfamily\color{BlackText},
    keywords={
        true,false,
        unsafe,async,await,move,
        use,pub,crate,super,self,mod,
        struct,enum,fn,const,static,let,mut,ref,type,impl,dyn,trait,where,as,
        break,continue,if,else,while,for,loop,match,return,yield,in
    },
    keywordstyle=\color{PurpleKeyword},
    ndkeywords={
        bool,u8,u16,u32,u64,u128,i8,i16,i32,i64,i128,char,str,
        Self,Option,Some,None,Result,Ok,Err,String,Box,Vec,Rc,Arc,Cell,RefCell,HashMap,BTreeMap,
        macro_rules
    },
    ndkeywordstyle=\color{RedTypename},
    comment=[l][\color{GrayComment}\slshape]{//},
    morecomment=[s][\color{GrayComment}\slshape]{/*}{*/},
    morecomment=[l][\color{GoldDocumentation}\slshape]{///},
    morecomment=[s][\color{GoldDocumentation}\slshape]{/*!}{*/},
    morecomment=[l][\color{GoldDocumentation}\slshape]{//!},
    morecomment=[s][\color{RedTypename}]{\#![}{]},
    morecomment=[s][\color{RedTypename}]{\#[}{]},
    stringstyle=\color{GreenString},
    string=[b]"
}

\lstset{literate=%
    {Ö}{{\"O}}1
    {Ä}{{\"A}}1
    {Ü}{{\"U}}1
    {ß}{{\ss}}1
    {ü}{{\"u}}1
    {ä}{{\"a}}1
    {ö}{{\"o}}1
    {~}{{\textasciitilde}}1
}

\usetheme{Montpellier}

\setbeamertemplate{title page}
{
  \vbox{}
  \begingroup
    \centering
    {
    {\small Bachelorarbeit}\\
    \begin{beamercolorbox}[sep=8pt,center]{title}
      \usebeamerfont{title}\inserttitle\par%
      \ifx\insertsubtitle\@empty%
      \else%
        \vskip0.25em%
        {\usebeamerfont{subtitle}\usebeamercolor[fg]{subtitle}\insertsubtitle\par}%
      \fi%
    \end{beamercolorbox}%
    \usebeamercolor[fg]{titlegraphic}\inserttitlegraphic\par}\vskip1em
    \vskip1em\par
    \begin{beamercolorbox}[sep=8pt,center]{author}
      \usebeamerfont{author}\insertauthor
      \begin{tabular}{ll}
          Verfasser & Bernd Haßfurther \href{mailto:nerglom@posteo.de}{<nerglom@posteo.de>}\\
          Matrikel-Nr. & 4372280\\
          Betreuerin & Prof. Dr. Lena Oden\\
          Datum & \today\\
      \end{tabular}
    \end{beamercolorbox}
    \begin{beamercolorbox}[sep=8pt,center]{institute}
      \usebeamerfont{institute}\insertinstitute
    \end{beamercolorbox}
  \endgroup
  \vfill
}

\addtobeamertemplate{navigation symbols}{}{%
    \usebeamerfont{footline}%
    \usebeamercolor[fg]{footline}%
    \hspace{1em}%
    \insertframenumber/\inserttotalframenumber
}

\setbeamertemplate{section in toc}[sections numbered]
\setbeamertemplate{subsection in toc}[subsections numbered]

\NewDocumentCommand{\codeword}{v}{%
\texttt{\textcolor{blue}{#1}}%
}

\title{Ein Computeralgebrasystem in Rust}
\titlegraphic{\includegraphics[width=2cm]{rust-logo-512x512-blk.png}}

\begin{document}
\maketitle
\frame{
  \frametitle{Inhaltsverzeichnis}
  \tableofcontents
}

\section{Vorstellung Computeralgebrasystem}
\begin{frame}[fragile]
  \frametitle{Was ist ein CAS?}
  \begin{itemize}
    \item Performance
    \pause
    \item Verlässlichkeit
    \pause
    \item Produktivität
  \end{itemize}
  {\small (vgl. \cite{WhyRust})}
\end{frame}

\begin{frame}[fragile]
  \frametitle{Funktionsumfang der Implementierung}
  \begin{itemize}
    \item Performance
    \pause
    \item Verlässlichkeit
    \pause
    \item Produktivität
  \end{itemize}
  {\small (vgl. \cite{WhyRust})}
\end{frame}

\section{Vorstellung Rust}
\begin{frame}[fragile]
  \frametitle{Ziele von Rust}
  \begin{itemize}
    \item Performance
    \pause
    \item Verlässlichkeit
    \pause
    \item Produktivität
  \end{itemize}
  {\small (vgl. \cite{WhyRust})}
\end{frame}

\begin{frame}[fragile]
  \frametitle{Syntax anhand von Beispielen}
  \begin{itemize}
    \item Performance
    \pause
    \item Verlässlichkeit
    \pause
    \item Produktivität
  \end{itemize}
  {\small (vgl. \cite{WhyRust})}
\end{frame}

\begin{frame}[fragile]
  \frametitle{Weitere Konzepte in Rust}
  \begin{itemize}
    \item Performance
    \pause
    \item Verlässlichkeit
    \pause
    \item Produktivität
  \end{itemize}
  {\small (vgl. \cite{WhyRust})}
\end{frame}

\begin{frame}[fragile]
  \frametitle{Vertiefung des Ownership und Borrowing}
  \begin{itemize}
    \item Performance
    \pause
    \item Verlässlichkeit
    \pause
    \item Produktivität
  \end{itemize}
  {\small (vgl. \cite{WhyRust})}
\end{frame}

\begin{frame}[fragile]
  \frametitle{Stack, Heap, Copy und Clone}
  \begin{itemize}
    \item Performance
    \pause
    \item Verlässlichkeit
    \pause
    \item Produktivität
  \end{itemize}
  {\small (vgl. \cite{WhyRust})}
\end{frame}

\section{Einen Term lesen}
\begin{frame}[fragile]
  \frametitle{Tokenizer}
  \begin{itemize}
    \item Performance
    \pause
    \item Verlässlichkeit
    \pause
    \item Produktivität
  \end{itemize}
  {\small (vgl. \cite{WhyRust})}
\end{frame}

\begin{frame}[fragile]
  \frametitle{Parser}
  \begin{itemize}
    \item Performance
    \pause
    \item Verlässlichkeit
    \pause
    \item Produktivität
  \end{itemize}
  {\small (vgl. \cite{WhyRust})}
\end{frame}

\section{Implementierung des CAS}
\begin{frame}[fragile]
  \frametitle{Überlauf und Ungenauigkeit}
  \begin{itemize}
    \item Performance
    \pause
    \item Verlässlichkeit
    \pause
    \item Produktivität
  \end{itemize}
  {\small (vgl. \cite{WhyRust})}
\end{frame}

\begin{frame}[fragile]
  \frametitle{Überlegungen zur Datenstruktur}
  \begin{itemize}
    \item Performance
    \pause
    \item Verlässlichkeit
    \pause
    \item Produktivität
  \end{itemize}
  {\small (vgl. \cite{WhyRust})}
\end{frame}

\begin{frame}[fragile]
  \frametitle{Vergleich zu existierenden Lösungen}
  \begin{itemize}
    \item Performance
    \pause
    \item Verlässlichkeit
    \pause
    \item Produktivität
  \end{itemize}
  {\small (vgl. \cite{WhyRust})}
\end{frame}

\begin{frame}[fragile]
  \frametitle{Datenstruktur in Rust}
  \begin{itemize}
    \item Performance
    \pause
    \item Verlässlichkeit
    \pause
    \item Produktivität
  \end{itemize}
  {\small (vgl. \cite{WhyRust})}
\end{frame}

\begin{frame}[fragile]
  \frametitle{Grundfunktionalitäten}
  \begin{itemize}
    \item Performance
    \pause
    \item Verlässlichkeit
    \pause
    \item Produktivität
  \end{itemize}
  {\small (vgl. \cite{WhyRust})}
\end{frame}

\begin{frame}[fragile]
  \frametitle{Erweiterung des CAS mit EvalFn}
  \begin{itemize}
    \item Performance
    \pause
    \item Verlässlichkeit
    \pause
    \item Produktivität
  \end{itemize}
  {\small (vgl. \cite{WhyRust})}
\end{frame}

\begin{frame}[fragile]
  \frametitle{Konkrete Erweiterungen des CAS}
  \begin{itemize}
    \item Performance
    \pause
    \item Verlässlichkeit
    \pause
    \item Produktivität
  \end{itemize}
  {\small (vgl. \cite{WhyRust})}
\end{frame}

\begin{frame}[fragile]
  \frametitle{Implementierung von mathematischen Funktionen}
  \begin{itemize}
    \item Performance
    \pause
    \item Verlässlichkeit
    \pause
    \item Produktivität
  \end{itemize}
  {\small (vgl. \cite{WhyRust})}
\end{frame}

\section{Vergleich zu SymPy}
\begin{frame}[fragile]
  \frametitle{Parsen von Termen}
  \begin{itemize}
    \item Performance
    \pause
    \item Verlässlichkeit
    \pause
    \item Produktivität
  \end{itemize}
  {\small (vgl. \cite{WhyRust})}
\end{frame}

\begin{frame}[fragile]
  \frametitle{Performance}
  \begin{itemize}
    \item Performance
    \pause
    \item Verlässlichkeit
    \pause
    \item Produktivität
  \end{itemize}
  {\small (vgl. \cite{WhyRust})}
\end{frame}

\section{Zusammenfassung und Fazit}
\begin{frame}[fragile]
  \frametitle{Verbesserungsideen und deren Ansätze}
  \begin{itemize}
    \item Performance
    \pause
    \item Verlässlichkeit
    \pause
    \item Produktivität
  \end{itemize}
  {\small (vgl. \cite{WhyRust})}
\end{frame}

\begin{frame}[fragile]
  \frametitle{Vor- und Nachteile Rust}
  \begin{itemize}
    \item Performance
    \pause
    \item Verlässlichkeit
    \pause
    \item Produktivität
  \end{itemize}
  {\small (vgl. \cite{WhyRust})}
\end{frame}

\section{Quellen}
\begin{frame}[fragile,allowframebreaks]
  \bibliographystyle{babplain-fl}
  \raggedright
  \setbeamertemplate{bibliography item}{\insertbiblabel}
  \bibliography{literature}
\end{frame}
\end{document}